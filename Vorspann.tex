%Sonderzeichen
\usepackage[T1]{fontenc}
\usepackage[utf8x]{inputenc} 

%Bilder
\usepackage{graphicx}
\usepackage{subfigure}
%Bild bennenen
\renewcommand{\figurename}{Bild}
%Nach Bild einrücken
\setlength{\parindent}{0pt}

%Stichwortverzeichnis
\usepackage{makeidx}
\makeindex

%Ränder
\usepackage{anysize}
\marginsize{2.4cm}{2.4cm}{2.2cm}{3cm}


%Sprache
\usepackage[german]{babel}
%\select@language{german}

%Numerierungstiefe
\setcounter {secnumdepth}{3}
\setcounter{tocdepth}{2}

%Verzeichnis verlinken
\usepackage{hyperref}

%Farben (Inhaltsverzeichnis)
\usepackage{color}
\definecolor{black}{rgb}{0,0,0}
\hypersetup{colorlinks, linkcolor=black}


%Kopf und Fusszeilen
\usepackage{fancyhdr}
\pagestyle{fancy} 								%vordefinierter Style
\fancyhf{}
%Kopfzeile
\fancyhead[L]{\nouppercase{\leftmark}}			%Oben Links Kapitel und Kapitelnummer
\fancyhead[C]{TG 13/3 Projektdokumentation}		%Oben Mitte
\fancyhead[R]{\today}							%Oben Rechts Datum
\renewcommand{\headrulewidth}{0.5pt}			%Strich oben
\headheight 15pt								%Abstand zum Seitenanfang
%Fusszeile
\fancyfoot[L]{Fabian Zeller }					%Unten Links
\fancyfoot[C]{\thepage}							%Unten Mitte Seitenzahl
\fancyfoot[R]{Emanuel Hubenschmidt}				%Unten Rechts
\renewcommand{\footrulewidth}{0.0pt}			%Strich unten
\footskip 63pt									%Abstand zum Seitenende

%Tabellen

\usepackage{tabularx}
\usepackage{float}
\floatplacement{figure}{H}

%Literaturverzeichnis
\usepackage{url}
\usepackage{cite}
\hypersetup{colorlinks, linkcolor=black}

%Mathematisch
\usepackage{amssymb}

%Code einbinden
\usepackage{listings} 
\usepackage{color}   
\usepackage[svgnames]{xcolor} 

\definecolor{mygreen}{rgb}{0,0.6,0}
\definecolor{mygray}{rgb}{0.5,0.5,0.5}
\definecolor{mymauve}{rgb}{0.58,0,0.82}

\lstset{language=Java}
\lstset{ %
  backgroundcolor=\color{white},   % choose the background color; you must add \usepackage{color} or \usepackage{xcolor}
  basicstyle=\footnotesize,        % the size of the fonts that are used for the code
  breakatwhitespace=false,         % sets if automatic breaks should only happen at whitespace
  breaklines=true,                 % sets automatic line breaking
  captionpos=b,                    % sets the caption-position to bottom
  commentstyle=\color{mygreen},    % comment style
  deletekeywords={...},            % if you want to delete keywords from the given language
  escapeinside={\%*}{*)},          % if you want to add LaTeX within your code
  extendedchars=true,              % lets you use non-ASCII characters; for 8-bits encodings only, does not work with UTF-8
  frame=single,	                   % adds a frame around the code
  keepspaces=true,                 % keeps spaces in text, useful for keeping indentation of code (possibly needs columns=flexible)
  keywordstyle=\color{blue},       % keyword style
  language=Java,                   % the language of the code
  otherkeywords={*,...},           % if you want to add more keywords to the set
  numbers=left,                    % where to put the line-numbers; possible values are (none, left, right)
  numbersep=5pt,                   % how far the line-numbers are from the code
  numberstyle=\tiny\color{mygray}, % the style that is used for the line-numbers
  rulecolor=\color{black},         % if not set, the frame-color may be changed on line-breaks within not-black text (e.g. comments (green here))
  showspaces=false,                % show spaces everywhere adding particular underscores; it overrides 'showstringspaces'
  showstringspaces=false,          % underline spaces within strings only
  showtabs=false,                  % show tabs within strings adding particular underscores
  stepnumber=2,                    % the step between two line-numbers. If it's 1, each line will be numbered
  stringstyle=\color{mymauve},     % string literal style
  tabsize=2,	                   % sets default tabsize to 2 spaces
  title=Codeausschnitt             % show the filename of files included with \lstinputlisting; also try caption instead of title
}

